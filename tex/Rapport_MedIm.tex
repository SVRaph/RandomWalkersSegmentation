\documentclass{article}

\usepackage[utf8]{inputenc} % accents
\usepackage[T1]{fontenc} % caractères français
\usepackage{geometry} % marges
\usepackage[french]{babel} % langue
\usepackage{graphicx} % images
\usepackage{verbatim} % texte préformaté
\usepackage{amsmath}
\usepackage{amsfonts} %mathbb

\newcommand{\R}{\mathbb{R}}

\title{Imagérie médicale - Projet \\ - \\ \small{Segmentation by retriveal with guided random walks: Application to left ventricle segmentation in MRI \\ A.Eslami A.Karamalis A.Katouzian N.Navab}} 
\author{Marie \textsc{Rozand}, Alexandre \textsc{Sarrazin} et Raphaël \textsc{Sivera}}
 
% \pagestyle{headings} % affiche un rappel discret (en haut à gauche)
% de la partie dans laquel on se situe

\begin{document}

\maketitle

\section{Introduction}

Dans cet article une nouvelle méthode de segmentation est présentée puis appliquée à la délimitation du ventricule gauche dans des séquences IRM.

La construction d'un a priori de forme statistique est parfois difficile et son utilisation peu adapté à la présence de cas extrême. Il s'agit donc ici d'utiliser directement la connaissance contenue dans une base de données d'images segmentées. Afin de garder un maximum de variabilité, ils utilisent cette base de données pour guider un algorithme de marche aléatoire.


\section{Prior work}

\section{Technical content}

\subsection{Guided random walk}

Si notre image contient des pixels déjà étiquettés (\textit{seeds}), la probabilité qu'une marche aléatoire partant d'un pixel atteigne un pixel avec tel ou tel label en premier définit une segmentation de l'image. Les probabilités de transition $w_{ij}$ entre deux pixels adjacents sont définies les caractéristiques de l'image (gradient de l'intensité, etc.).

On peut montrer le calcul des probabilités peut se ramener à un problème d'optimisation quadratique de l'énergie : 
$$ E(x)= \frac{1}{2} \sum_{i,j}{w_{ij} (x_i -x_j)^2} $$

S'il n'est pas facile de modifier directement les probabilités pour tenir compte d'une connaissance a priori, il est aisé de modifier la formulation variationnelle. On cherche donc à miniser :
$$ E(x)= \frac{1}{2} \sum_{i,j}{w_{ij} (x_i -x_j)^2} + \frac{\gamma}{2} \sum_{i,j}{\omega_{ij} (x_i -b_j)^2}$$

où l'on définit $I_i$ l'intensité de l'image au pixel $i$, $R_i$ l'intensité de l'image guide en $i$, $b_i$ le label du pixel $i$ de l'image $R$ ainsi que :
\begin{align*}
w_{ij} &= \exp(-\beta(I_i-I_j)^2) \text{ si $j$ est dans le voisinage intra-image de $i$} \\
      &= 0 \text{ sinon}\\
\omega_{ij} &= \exp(-\alpha(I_i-R_j)^2) \text{ si $j$ est dans le voisinage inter-images de $i$}\\
           &= 0 \text{ sinon}\\
\end{align*}

La solution à ce problème de minimisation est explicite et revient à résoudre un système linéaire parcimonieux.

\subsection{Segmentation by retrival}

L'algorithme consiste à appliquer l'algorithme de \textit{guided random walk} pour chaque image présente dans la database. Pour sélectionner la segmentation à garder on compare la segmentation obtenue en seuillant le champs de probabilité $x$ à la segmentation de l'image guide. Si la distance de Dice est minimale cela signifie que les deux images se ressemblent, on garde alors la segmentation obtenue avec cette référence.


\section{Résultats}

Cette méthode dépend de 5 paramètres : $\alpha$, $\beta$, $\gamma$, du seuil utilisé pour segmenté et du nombre de seeds. Pour optimiser la méthode nous avons étudié l’efficacité de la segmentation en faisant varier ces paramètres.

\subsection{Paramètre de la marche aléatoire}

Tout d’abord, il y a trois paramètres qui contrôlent l’algorithme de marche aléatoire guidée : $\alpha$, $\beta$ et $\gamma$. 

Le paramètre $\gamma$ jauge de l’influence des images de la base d’image segmentée pour segmenter la nouvelle image. Si $\gamma=0$, l’algorithme se résume à une simple marche aléatoire. Plus il est grand, plus l’influence des connaissances a priori sont prises en compte. 

Le paramètre $\alpha$ définit la sensibilité à la variation d’intensité au sein de chaque classe dans l’image à segmenter. 

Le paramètre $\beta$ définit la sensibilité de la méthode à la différence d’intensité entre l’image à segmentée et celles de la base de données. 

Les valeurs de ces paramètres qui optimisent la méthode ont été évaluées de manière empirique. On évalue la performance pour un patient et une base de donnée fixe et on représente la mesure de Dice pour chaque valeur de paramètre.


Le graphe (a) représente la précision de la segmentation pour différentes valeurs de γ et à α=10 et β=30 fixés. On voit que le γ est optimal aux alentours de 10. Cet optimum a été évalué sur la segmentation d’une seule image alors que d’en l’article ils faisaient la moyenne sur 6 images. Cependant l’efficacité aux environs de γ=10 est assez stable (99\% du maximum entre 1 et 30). Par contre, dans l’article, ils conseillent de prendre la valeur 0.4 qui est reste assez efficace dans notre cas (Dice=0.88) mais qui est quand même relativement éloignée de notre optimum. Cette différence s’explique par la dépendance du paramètre γ  avec la base de données utilisée. Si elle contient une image très proche de l’image à segmenter alors, il est évidemment préférable de prendre fortement en compte l’information de l’image de la base et donc d’utiliser un γ élevé. Comme nos images sont synthétiques, leur variabilité est plus faible que dans les données réelles utilisées dans l’article, d’où la différence d’optimum. Ce paramètre est donc assez difficile à estimer a priori et dépend fortement de la richesse de la base de données par rapport à la variabilité des images.

 Le graphe (b) représente la mesure de Dice obtenue pour différentes valeur de α, à β=30 et γ=0.4 fixés. La valeur optimale que l’on obtient est pour α=15 mais on remarque qu’il suffit de prendre une valeur suffisamment grande pour que l’algorithme fonctionne. En effet, si on prend α trop petit, l’algorithme compare les labels des pixels voisins même si ceux-ci ont des intensités très différentes. L’article montre une courbe très proche mais avec un α quatre fois plus grand. La valeur de α dépend de l’intervalle dans lequel sont prises les intensités. Dans nos données synthétiques nous avons des intensités comprises entre [-1,1]. Nous en avons donc déduis que leurs intensités devaient être dans un intervalle deux fois plus petit [0,1] ce qui est standard dans le traitement d’image. Cela justifierait la nécessité d’un paramètre 4 fois plus grand dans le poids gaussien.


Le graphe (c) illustre la précision de la méthode suivant le paramètre β et à α=10 et γ=0.4 fixés. De même que pour α, on observe que l’optimum et à 20 mais que l’efficacité décroit très faiblement après cette valeur de paramètre. Il faut donc simplement prendre une valeur suffisamment grande pour pénaliser des arêtes entre deux voxels d’intensité très différentes et ne pas considérer le label du voxel de l’image de la base s’il n’est pas d’intensité similaire au voxel de l’image à segmenter. De plus on obtient aussi des valeurs différentes à celles de l’article, toujours à causes de la différence entre les intervalles d’intensités.



%À $\alpha=90$ et $\beta=1.3$ fixés, on voit que le $\gamma$ est optimal aux alentours de 0.3. La précision de la segmentation varie entre 0.2 et 0.4 avec une chute importante pour les valeurs inférieures, nous avons donc choisi de fixer $\gamma=0.4$ pour la suite de notre évaluation.

\subsection{Seuil de segmentation}

L'algorithme de marche aléatoire renvoie une carte de probabilités. Il faut seuiller ces valeurs pour obtenir une segmentation. Les auteurs proposent simplement de seuiller à $0,5$ : un voxel appartient à l'objet si un marcheur aléatoire partant de ce point à plus de chance d'atteindre en premier une graine de l'objet et inversement. Cependant ce seuil ne tient pas compte des a priori que l'on pourrait avoir sur l'objet. Un voxel a par exemple a priori plus de chance d'appartenir à l'arrière plan qui occupe une majorité du volume. Cela ne tient pas compte de la répartition des seeds. D'ailleurs expérimentalement ce seuil ne nous convient pas. On obtient de meilleurs résultats (mesure de Dice) pour un seuil au alentour de $0,4$~\ref{fig:seg_threshold}. Comme on le voit sur la figure ~\ref{fig:seg_threshold}, le résultat est alors peu sensible à ce seuil et il est possible de l'adapter en fonction de l'évolution du volume de l'objet segmenté. En effet lorsque l'on se rapproche de la valeur optimale le volume se met à diminuer de manière importante.
\begin{center}
\begin{figure}[h]
\label{fig:seg_threshold}
 \caption{Évolution de la similarité en fonction du seuil de segmentation $\alpha=15$, $\beta=25$, $\gamma=4$}
\includegraphics[width=0.8\textwidth]{../fig/seg_threshold_alpha15_beta25_gamma4.png}
\end{figure}
\end{center}


\subsection{Nombre de graines}

Ensuite, nous avons étudié l’influence du nombre de graines sur la précision de la segmentation. Elles permettent de guider la marche aléatoire mais aussi de cibler la cardioïde à segmenter. L’algorithme compare les images pixels par pixels et est donc très sensible aux décalages des images. Il faut donc centrer toute les images sur les cardioïdes pour maximiser la similarité de l’image avec la base de données. Dans notre cas 



maxDice pour alpha =
    0.5447	0.1
    0.5486	0.3
    0.5596	1
    0.5799	3
    0.5892	10
    0.5649	50
    0.5582	80
    0.5553	100
    0.5495	150
    0.5462	200
    0.5422	300




\section{Critiques}

\section{Travaux ultérieurs}

\section{Conclusion}



\end{document}
