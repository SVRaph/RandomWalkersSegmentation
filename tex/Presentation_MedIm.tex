\documentclass{beamer}
\usepackage[utf8]{inputenc}
%\usepackage{subfig} % sous-figures
%\usepackage{subfigure}
\usepackage{xcolor} % couleurs
\usepackage{amssymb} % symboles maths
% \usepackage{media9} % fichiers audio
\usetheme{Frankfurt}
\addtobeamertemplate{footline}{\insertframenumber/\inserttotalframenumber}

\graphicspath{{../fig/}}
\usepackage{multimedia}

\def\var{{\rm Var}}
\def\cov{{\rm cov}} 
\newcommand\RR{\mathbb{R}} 

\title{Imagérie médicale - Projet \\ - \\ \small{Segmentation by retriveal with guided random walks: Application to left ventricle segmentation in MRI \\ A.Eslami A.Karamalis A.Katouzian N.Navab}} 
\author{Marie \textsc{Rozand} et Raphaël \textsc{Sivera}}




% \subfloat{\includegraphics[scale=0.25]{logo_enpc.jpg}} 
\begin{document}
\begin{frame}
  \begin{center}

    \titlepage

  \end{center}
\end{frame}


\begin{frame}{Plan}
  \tableofcontents
\end{frame}



\section{Introduction}
\begin{frame}{Objectif}
  \textbf{Objectif : }Segmenter le ventricule gauche sur des séquences IRM (a) en utilisant une base de donnée présegmentée (b) et (c).
  \begin{figure}[h]
    \begin{center}
      \includegraphics[width=0.8\textwidth]{coeurs.png}
    \end{center}
  \end{figure}
\end{frame}


\begin{frame}{Dataset synthétique}
On génère un dataset où les ventricules sont modélisés par des ellipsoïdes. 
\begin{figure}[h]
  \begin{center}
    \includegraphics[width=0.6\textwidth]{seeds.png}
  \end{center}
\end{figure}
\end{frame}


\section{Méthode proposée}
\begin{frame}{Marche aléatoire}
  \begin{itemize}
  \item L'image définit un graphe : un voxel $\leftrightarrow$ un sommet.
  \item Le poids d'une arète en deux voxels adjacents est lié au gradient d'intensité.
  \item La carte de probabilité qu'une marche aléatoire partant du voxel passe d'abord pas un voxel avec telle ou telle étiquette définie une segmentation
  \end{itemize}
\end{frame}

\begin{frame}{Marche aléatoire guidée}
Énergie associée à une marche aléatoire \only<2>{guidée} :
\only<1>{$$ E(x)= \frac{1}{2} \sum_{i,j}{w_{ij} (x_i -x_j)^2} $$}
\only<2>{$$ E(x)= \frac{1}{2} \sum_{i,j}{w_{ij} (x_i -x_j)^2} + \frac{\gamma}{2} \sum_{i,j}{\omega_{ij} (x_i -b_j)^2}$$}

pour $w_{ij} = \exp(-\alpha(I_i-I_j)^2)$ et $\omega_{ij} = \exp(-\beta(I_i-R_j)^2)$

Résolution d'un système linéaire sparse.
\end{frame}


\begin{frame}{Segmentation par sélection}
\begin{itemize}
\item Appliquer l'algorithme de \textit{guided random walk} pour chaque image de la database
\item Sélectionner la segmentation en la comparant à la segmentation obtenue
\item Si aucune image ne convient, effectuer une marche aléatoire non-guidée.
\end{itemize}
\end{frame}


\section{Résultats}
\begin{frame}{Résultats}
  \begin{figure}[h]
    \begin{center}
      \parbox{5cm}{\includegraphics[width=0.45\textwidth]{../fig/segmentation_result_true.png}}
      \parbox{5cm}{\includegraphics[width=0.45\textwidth]{../fig/segmentation_result_driver.png}}
      \caption{Segmentation obtenue pour $\alpha=15$, $\beta=25$, $\gamma=4$ (en vert) et Segmentation de référence ou segmentation guide (en rouge), Dice : 0.93}
    \end{center}
  \end{figure}
\end{frame}

%\section{Résultats}
%\begin{frame}{Image 3D}
%  \begin{figure}[h]
%    \begin{center}
%\movie[width=\textwidth,showcontrols=true]
%{\includegraphics[scale=1]{segmentation_result_driver.png}}{../fig/anim.avi} \\
%    \end{center}
%  \end{figure}
%\end{frame}




\begin{frame}{Paramètres de la marche aléatoire}

3 paramètres contrôlent la marche aléatoire guidée :  
$$ E(x)= \frac{1}{2} \sum_{i,j}{w_{ij} (x_i -x_j)^2} + \frac{\gamma}{2} \sum_{i,j}{\omega_{ij} (x_i -b_j)^2}$$
pour $w_{ij} = \exp(-\alpha(I_i-I_j)^2)$ et $\omega_{ij} = \exp(-\beta(I_i-R_j)^2)$

\begin{itemize}
\item $\gamma$ : influence des images de la base d’image segmentée pour segmenter la nouvelle image. 

\item $\alpha$ : sensibilité à la variation d’intensité dans l’image à segmenter.

\item $\beta$ : sensibilité à la différence d’intensité entre l’image à segmentée et celles de la base de données. 
\end{itemize}

\end{frame}


\begin{frame}{Paramètres de la marche aléatoire - $\gamma$}
%\framesubtitle{$\gamma$}
\begin{figure}[h!]
  \begin{center}
    \caption{Évolution de la similarité en fonction de $\gamma$ pour $\alpha=15$, $\beta=25$}
    \includegraphics[width=0.8\textwidth]{../fig/gamma_alpha15_beta25.png}
  \end{center}
\end{figure}
\end{frame}

\begin{frame}{Paramètres de la marche aléatoire - $\alpha \beta$}
  \begin{figure}[h]
    \begin{center}
      \parbox{5cm}{\includegraphics[width=0.5\textwidth]{../fig/alpha_beta30_gamma0,4.png}}
      \parbox{5cm}{\includegraphics[width=0.5\textwidth]{../fig/beta_alpha10_gamma0,4.png}}
      \caption{Évolution de la similarité en fonction de $\alpha$ pour  $\beta=30$ et de $\beta$ pour $\alpha=10$, $\gamma=0.4$}
    \end{center}
  \end{figure}
\end{frame}


\begin{frame}{Paramètres de la marche aléatoire - Nombre de graines}
\begin{figure}[h!]
  \begin{center}
    \caption{Évolution de la similarité en fonction du nombre de graines, $\alpha=10$, $\beta=20$, $\gamma=4$}
    \includegraphics[width=0.8\textwidth]{../fig/nseeds_alpha15_beta20_gamma4.png}
    \label{fig:nseeds}
  \end{center}
\end{figure}
\end{frame}



\section{Critiques}
\subsection*{Seuil de segmentation}
\begin{frame}{Seuil de segmentation}
\begin{figure}[h]
  \begin{center}
    \caption{Évolution de la similarité en fonction du seuil de segmentation $\alpha=15$, $\beta=25$, $\gamma=4$}
    \includegraphics[width=0.8\textwidth]{../fig/seg_threshold_alpha15_beta25_gamma4.png}
    \label{fig:seg_threshold}
  \end{center}
\end{figure}
\end{frame}

\begin{frame}{Seuil de segmentation}
\begin{itemize}
\item Pas étudier dans l'article, seuil=0.5
\item A priori et asymétrie de l'image
\item Compromis sensibilité-spécificité
\end{itemize}
\begin{figure}[h!]
  \begin{center}
    \caption{courbe ROC}
    \includegraphics[width=0.6\textwidth]{../fig/ROC.png}
  \end{center}
\end{figure}
\end{frame}

\subsection{Critères d'évaluation}
\begin{frame}{Critères d'évaluation}
\begin{itemize}
\item Dice $2 \| B_0 \cap B_1 \| / ( \| B_0 \| + \| B_1 \| )$ 
\item \textit{Shape similarity} $ S(X,B)= \frac{1}{\|\partial B \|} \sum_{n \in \partial B}{ \left( \frac{\|\nabla L_x(n) \cdot \nabla L_B(n) \|}{\|\nabla L_x(n)\| \| \nabla L_B(n) \|} \right)} $ où $L_A$ correspond à la distance signée à la frontière de $A$.
\item Déplacement de la frontière
\item Erreur sur le volume des sous-parties du cœur...
\end{itemize}
\end{frame}

\begin{frame}{Critères d'évaluation}
\begin{figure}[h]
  \begin{center}
    \caption{\textit{Shape similarity} et Mesure de Dice}
      \includegraphics[width=0.8\textwidth]{../fig/Shape_similarity_Dice.png}
      \label{fig:similarities}
  \end{center}
\end{figure}
\end{frame}


\section{Conclusion}
\begin{frame}{Conclusion - Perspectives}
\begin{itemize}
\item Exploite efficacement les relations inter-images.
\item Les paramètres $\gamma$ et $\beta$ permettent de tenir compte de la variabilité du dataset.
\item Sélection automatique du meilleur modèle.
\item Meilleur qu'une marche aléatoire.
\end{itemize}

 \vspace{0.5cm}
  $\Rightarrow$ Problème du recalage géométrique et dynamique\\
  $\Rightarrow$ Comparaison au niveau du voxel. \\
  $\Rightarrow$ Une exécution pour chaque image de référence.
\end{frame}
%%%%%%%%%%%%%%%%%%%%%%%%%%%%%%%%%%%%%%%%%%%%%%%%%%%%%%%%%%%%%%%%%%%%%%%%%% 
%%%%%%%%%%%%%%%%%%%%%%%%%%%%%%%%%%%%%%%%%%%%%%%%%%%%%%%%%%%%%%%%%%%%%%%%%% 
\section{}

\begin{frame}[noframenumbering]
  \frametitle{C'est fini}
\end{frame}


\end{document}
